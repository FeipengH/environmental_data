% Options for packages loaded elsewhere
\PassOptionsToPackage{unicode}{hyperref}
\PassOptionsToPackage{hyphens}{url}
%
\documentclass[
]{article}
\usepackage{amsmath,amssymb}
\usepackage{lmodern}
\usepackage{iftex}
\ifPDFTeX
  \usepackage[T1]{fontenc}
  \usepackage[utf8]{inputenc}
  \usepackage{textcomp} % provide euro and other symbols
\else % if luatex or xetex
  \usepackage{unicode-math}
  \defaultfontfeatures{Scale=MatchLowercase}
  \defaultfontfeatures[\rmfamily]{Ligatures=TeX,Scale=1}
\fi
% Use upquote if available, for straight quotes in verbatim environments
\IfFileExists{upquote.sty}{\usepackage{upquote}}{}
\IfFileExists{microtype.sty}{% use microtype if available
  \usepackage[]{microtype}
  \UseMicrotypeSet[protrusion]{basicmath} % disable protrusion for tt fonts
}{}
\makeatletter
\@ifundefined{KOMAClassName}{% if non-KOMA class
  \IfFileExists{parskip.sty}{%
    \usepackage{parskip}
  }{% else
    \setlength{\parindent}{0pt}
    \setlength{\parskip}{6pt plus 2pt minus 1pt}}
}{% if KOMA class
  \KOMAoptions{parskip=half}}
\makeatother
\usepackage{xcolor}
\usepackage[margin=1in]{geometry}
\usepackage{color}
\usepackage{fancyvrb}
\newcommand{\VerbBar}{|}
\newcommand{\VERB}{\Verb[commandchars=\\\{\}]}
\DefineVerbatimEnvironment{Highlighting}{Verbatim}{commandchars=\\\{\}}
% Add ',fontsize=\small' for more characters per line
\usepackage{framed}
\definecolor{shadecolor}{RGB}{248,248,248}
\newenvironment{Shaded}{\begin{snugshade}}{\end{snugshade}}
\newcommand{\AlertTok}[1]{\textcolor[rgb]{0.94,0.16,0.16}{#1}}
\newcommand{\AnnotationTok}[1]{\textcolor[rgb]{0.56,0.35,0.01}{\textbf{\textit{#1}}}}
\newcommand{\AttributeTok}[1]{\textcolor[rgb]{0.77,0.63,0.00}{#1}}
\newcommand{\BaseNTok}[1]{\textcolor[rgb]{0.00,0.00,0.81}{#1}}
\newcommand{\BuiltInTok}[1]{#1}
\newcommand{\CharTok}[1]{\textcolor[rgb]{0.31,0.60,0.02}{#1}}
\newcommand{\CommentTok}[1]{\textcolor[rgb]{0.56,0.35,0.01}{\textit{#1}}}
\newcommand{\CommentVarTok}[1]{\textcolor[rgb]{0.56,0.35,0.01}{\textbf{\textit{#1}}}}
\newcommand{\ConstantTok}[1]{\textcolor[rgb]{0.00,0.00,0.00}{#1}}
\newcommand{\ControlFlowTok}[1]{\textcolor[rgb]{0.13,0.29,0.53}{\textbf{#1}}}
\newcommand{\DataTypeTok}[1]{\textcolor[rgb]{0.13,0.29,0.53}{#1}}
\newcommand{\DecValTok}[1]{\textcolor[rgb]{0.00,0.00,0.81}{#1}}
\newcommand{\DocumentationTok}[1]{\textcolor[rgb]{0.56,0.35,0.01}{\textbf{\textit{#1}}}}
\newcommand{\ErrorTok}[1]{\textcolor[rgb]{0.64,0.00,0.00}{\textbf{#1}}}
\newcommand{\ExtensionTok}[1]{#1}
\newcommand{\FloatTok}[1]{\textcolor[rgb]{0.00,0.00,0.81}{#1}}
\newcommand{\FunctionTok}[1]{\textcolor[rgb]{0.00,0.00,0.00}{#1}}
\newcommand{\ImportTok}[1]{#1}
\newcommand{\InformationTok}[1]{\textcolor[rgb]{0.56,0.35,0.01}{\textbf{\textit{#1}}}}
\newcommand{\KeywordTok}[1]{\textcolor[rgb]{0.13,0.29,0.53}{\textbf{#1}}}
\newcommand{\NormalTok}[1]{#1}
\newcommand{\OperatorTok}[1]{\textcolor[rgb]{0.81,0.36,0.00}{\textbf{#1}}}
\newcommand{\OtherTok}[1]{\textcolor[rgb]{0.56,0.35,0.01}{#1}}
\newcommand{\PreprocessorTok}[1]{\textcolor[rgb]{0.56,0.35,0.01}{\textit{#1}}}
\newcommand{\RegionMarkerTok}[1]{#1}
\newcommand{\SpecialCharTok}[1]{\textcolor[rgb]{0.00,0.00,0.00}{#1}}
\newcommand{\SpecialStringTok}[1]{\textcolor[rgb]{0.31,0.60,0.02}{#1}}
\newcommand{\StringTok}[1]{\textcolor[rgb]{0.31,0.60,0.02}{#1}}
\newcommand{\VariableTok}[1]{\textcolor[rgb]{0.00,0.00,0.00}{#1}}
\newcommand{\VerbatimStringTok}[1]{\textcolor[rgb]{0.31,0.60,0.02}{#1}}
\newcommand{\WarningTok}[1]{\textcolor[rgb]{0.56,0.35,0.01}{\textbf{\textit{#1}}}}
\usepackage{graphicx}
\makeatletter
\def\maxwidth{\ifdim\Gin@nat@width>\linewidth\linewidth\else\Gin@nat@width\fi}
\def\maxheight{\ifdim\Gin@nat@height>\textheight\textheight\else\Gin@nat@height\fi}
\makeatother
% Scale images if necessary, so that they will not overflow the page
% margins by default, and it is still possible to overwrite the defaults
% using explicit options in \includegraphics[width, height, ...]{}
\setkeys{Gin}{width=\maxwidth,height=\maxheight,keepaspectratio}
% Set default figure placement to htbp
\makeatletter
\def\fps@figure{htbp}
\makeatother
\setlength{\emergencystretch}{3em} % prevent overfull lines
\providecommand{\tightlist}{%
  \setlength{\itemsep}{0pt}\setlength{\parskip}{0pt}}
\setcounter{secnumdepth}{-\maxdimen} % remove section numbering
\ifLuaTeX
  \usepackage{selnolig}  % disable illegal ligatures
\fi
\IfFileExists{bookmark.sty}{\usepackage{bookmark}}{\usepackage{hyperref}}
\IfFileExists{xurl.sty}{\usepackage{xurl}}{} % add URL line breaks if available
\urlstyle{same} % disable monospaced font for URLs
\hypersetup{
  pdftitle={Lab1.R},
  pdfauthor={stonehuang},
  hidelinks,
  pdfcreator={LaTeX via pandoc}}

\title{Lab1.R}
\author{stonehuang}
\date{2022-09-21}

\begin{document}
\maketitle

\begin{Shaded}
\begin{Highlighting}[]
\CommentTok{\#Lab 1}
\CommentTok{\#Feipeng Huang}

\FunctionTok{c}\NormalTok{(}\DecValTok{1}\NormalTok{, }\DecValTok{2}\NormalTok{, }\DecValTok{3}\NormalTok{)}
\end{Highlighting}
\end{Shaded}

\begin{verbatim}
## [1] 1 2 3
\end{verbatim}

\begin{Shaded}
\begin{Highlighting}[]
\StringTok{"c(1, 2, 3)"}
\end{Highlighting}
\end{Shaded}

\begin{verbatim}
## [1] "c(1, 2, 3)"
\end{verbatim}

\begin{Shaded}
\begin{Highlighting}[]
\CommentTok{\#Q1}
\CommentTok{\#c(1, 2, 3) is numeric, "c(1, 2, 3)" is character}

\NormalTok{c\_1 }\OtherTok{=} \FunctionTok{c}\NormalTok{(}\DecValTok{1}\NormalTok{, }\DecValTok{2}\NormalTok{, }\DecValTok{3}\NormalTok{)}
\NormalTok{c\_2 }\OtherTok{=} \StringTok{"c(1, 2, 3)"}

\CommentTok{\#Q2}
\CommentTok{\#function, c\_1 equals to a vector (no quotation marks)}

\CommentTok{\#Q3}
\CommentTok{\#variable, c\_2 equals to a character/string (has quotation marks)}

\CommentTok{\#Q4}
\CommentTok{\#c\_1 equals to a vector with elements 1, 2, and 3}
\CommentTok{\#c\_2 equals to a character called "c(1, 2, 3)"}

\NormalTok{my\_vec }\OtherTok{=} \DecValTok{1}\SpecialCharTok{:}\DecValTok{6}
\NormalTok{mat\_1 }\OtherTok{=} \FunctionTok{matrix}\NormalTok{(my\_vec, }\AttributeTok{nrow =} \DecValTok{3}\NormalTok{)}
\NormalTok{mat\_1}
\end{Highlighting}
\end{Shaded}

\begin{verbatim}
##      [,1] [,2]
## [1,]    1    4
## [2,]    2    5
## [3,]    3    6
\end{verbatim}

\begin{Shaded}
\begin{Highlighting}[]
\CommentTok{\#Q5}
\CommentTok{\#The matrix has 3 rows and 2 columns.}

\CommentTok{\#Q6}
\NormalTok{mat\_1[}\DecValTok{3}\NormalTok{,}\DecValTok{1}\NormalTok{]}
\end{Highlighting}
\end{Shaded}

\begin{verbatim}
## [1] 3
\end{verbatim}

\begin{Shaded}
\begin{Highlighting}[]
\CommentTok{\#Q7}
\NormalTok{mat\_2 }\OtherTok{=} \FunctionTok{matrix}\NormalTok{(my\_vec, }\AttributeTok{nrow =} \DecValTok{2}\NormalTok{, }\AttributeTok{ncol =} \DecValTok{3}\NormalTok{)}
\NormalTok{mat\_2}
\end{Highlighting}
\end{Shaded}

\begin{verbatim}
##      [,1] [,2] [,3]
## [1,]    1    3    5
## [2,]    2    4    6
\end{verbatim}

\begin{Shaded}
\begin{Highlighting}[]
\CommentTok{\#Q8}
\NormalTok{mat\_3 }\OtherTok{=} \FunctionTok{matrix}\NormalTok{(my\_vec, }\AttributeTok{nrow =} \DecValTok{3}\NormalTok{, }\AttributeTok{ncol =} \DecValTok{2}\NormalTok{)}
\NormalTok{mat\_3}
\end{Highlighting}
\end{Shaded}

\begin{verbatim}
##      [,1] [,2]
## [1,]    1    4
## [2,]    2    5
## [3,]    3    6
\end{verbatim}

\begin{Shaded}
\begin{Highlighting}[]
\CommentTok{\#Q9}
\CommentTok{\# R uses columns to recycle/distribute the values.}

\CommentTok{\#Q10}
\NormalTok{mat\_4 }\OtherTok{=} \FunctionTok{matrix}\NormalTok{(my\_vec, }\AttributeTok{nrow =} \DecValTok{2}\NormalTok{, }\AttributeTok{ncol =} \DecValTok{2}\NormalTok{)}
\NormalTok{mat\_4}
\end{Highlighting}
\end{Shaded}

\begin{verbatim}
##      [,1] [,2]
## [1,]    1    3
## [2,]    2    4
\end{verbatim}

\begin{Shaded}
\begin{Highlighting}[]
\CommentTok{\#Q11}
\CommentTok{\# R still uses columns to recycle/distribute the values.}

\NormalTok{first }\OtherTok{=} \FloatTok{5.2}
\NormalTok{second }\OtherTok{=} \StringTok{"five point two"}
\NormalTok{third }\OtherTok{=} \DecValTok{0}\SpecialCharTok{:}\DecValTok{5}
\NormalTok{my\_list\_1 }\OtherTok{=} \FunctionTok{list}\NormalTok{(first, second, third)}
\NormalTok{my\_list\_1}
\end{Highlighting}
\end{Shaded}

\begin{verbatim}
## [[1]]
## [1] 5.2
## 
## [[2]]
## [1] "five point two"
## 
## [[3]]
## [1] 0 1 2 3 4 5
\end{verbatim}

\begin{Shaded}
\begin{Highlighting}[]
\FunctionTok{names}\NormalTok{(my\_list\_1) }\OtherTok{=} \FunctionTok{c}\NormalTok{(}\StringTok{"two"}\NormalTok{, }\StringTok{"one"}\NormalTok{, }\StringTok{"three"}\NormalTok{)}
\NormalTok{my\_list\_1}
\end{Highlighting}
\end{Shaded}

\begin{verbatim}
## $two
## [1] 5.2
## 
## $one
## [1] "five point two"
## 
## $three
## [1] 0 1 2 3 4 5
\end{verbatim}

\begin{Shaded}
\begin{Highlighting}[]
\CommentTok{\#Q12}
\NormalTok{my\_list\_1[[}\DecValTok{1}\NormalTok{]] }
\end{Highlighting}
\end{Shaded}

\begin{verbatim}
## [1] 5.2
\end{verbatim}

\begin{Shaded}
\begin{Highlighting}[]
\CommentTok{\#value}
\CommentTok{\#by position}
\CommentTok{\#select the first element of the list}
\NormalTok{my\_list\_1[[}\FunctionTok{as.numeric}\NormalTok{(}\StringTok{"1"}\NormalTok{)]] }
\end{Highlighting}
\end{Shaded}

\begin{verbatim}
## [1] 5.2
\end{verbatim}

\begin{Shaded}
\begin{Highlighting}[]
\CommentTok{\#value}
\CommentTok{\#by position}
\CommentTok{\#select the first element of the list}
\NormalTok{my\_list\_1[[}\StringTok{"1"}\NormalTok{]] }
\end{Highlighting}
\end{Shaded}

\begin{verbatim}
## NULL
\end{verbatim}

\begin{Shaded}
\begin{Highlighting}[]
\CommentTok{\#NULL}
\CommentTok{\#By name}
\CommentTok{\#select the element called "1" in the list}
\NormalTok{my\_list\_1[[}\StringTok{"one"}\NormalTok{]] }
\end{Highlighting}
\end{Shaded}

\begin{verbatim}
## [1] "five point two"
\end{verbatim}

\begin{Shaded}
\begin{Highlighting}[]
\CommentTok{\#value}
\CommentTok{\#by name}
\CommentTok{\#select the element called "one" in the list}
\NormalTok{my\_list\_1}\SpecialCharTok{$}\NormalTok{one}
\end{Highlighting}
\end{Shaded}

\begin{verbatim}
## [1] "five point two"
\end{verbatim}

\begin{Shaded}
\begin{Highlighting}[]
\CommentTok{\#value}
\CommentTok{\#by name}
\CommentTok{\#select the element called "one" in the list}
\NormalTok{my\_list\_1}\SpecialCharTok{$}\StringTok{"one"}
\end{Highlighting}
\end{Shaded}

\begin{verbatim}
## [1] "five point two"
\end{verbatim}

\begin{Shaded}
\begin{Highlighting}[]
\CommentTok{\#value}
\CommentTok{\#by name}
\CommentTok{\#select the element called "one" in the list}
\CommentTok{\#my\_list\_1$1}
\CommentTok{\#error}
\CommentTok{\#by position but missing [[]] so not valid}
\NormalTok{my\_list\_1}\SpecialCharTok{$}\StringTok{"1"}
\end{Highlighting}
\end{Shaded}

\begin{verbatim}
## NULL
\end{verbatim}

\begin{Shaded}
\begin{Highlighting}[]
\CommentTok{\#NULL}
\CommentTok{\#by name}
\CommentTok{\#select the element called "1" in the list}

\CommentTok{\#Q13}
\NormalTok{my\_list\_1[[}\StringTok{"one"}\NormalTok{]] }
\end{Highlighting}
\end{Shaded}

\begin{verbatim}
## [1] "five point two"
\end{verbatim}

\begin{Shaded}
\begin{Highlighting}[]
\NormalTok{my\_list\_1}\SpecialCharTok{$}\NormalTok{one}
\end{Highlighting}
\end{Shaded}

\begin{verbatim}
## [1] "five point two"
\end{verbatim}

\begin{Shaded}
\begin{Highlighting}[]
\NormalTok{my\_list\_1}\SpecialCharTok{$}\StringTok{"one"}
\end{Highlighting}
\end{Shaded}

\begin{verbatim}
## [1] "five point two"
\end{verbatim}

\begin{Shaded}
\begin{Highlighting}[]
\CommentTok{\#They all select the element called "one", which is "five point two".}

\CommentTok{\#Q14}
\NormalTok{my\_list\_1[[}\StringTok{"1"}\NormalTok{]] }
\end{Highlighting}
\end{Shaded}

\begin{verbatim}
## NULL
\end{verbatim}

\begin{Shaded}
\begin{Highlighting}[]
\NormalTok{my\_list\_1}\SpecialCharTok{$}\StringTok{"1"}
\end{Highlighting}
\end{Shaded}

\begin{verbatim}
## NULL
\end{verbatim}

\begin{Shaded}
\begin{Highlighting}[]
\CommentTok{\#They both select the element called "1", which does not exist in the list.}
\end{Highlighting}
\end{Shaded}


\end{document}
